\documentclass[12pt]{amsart}
\usepackage{mathrsfs}
\usepackage{amsmath}
\usepackage{amssymb}
\usepackage{amsfonts}
\usepackage{amsopn}
\usepackage{amsthm}
\usepackage{latexsym}
\usepackage[all]{xy}
\usepackage{enumerate}
\usepackage{geometry}
%\usepackage{biblatex}
%\usepackage{hyperref}
%\usepackage[autostyle]{csquotes}
\usepackage{fancyhdr}
\usepackage{graphicx}
\usepackage{wrapfig}
\usepackage{float}

\usepackage[
    backend=biber,
style=alphabetic,
sorting=nyt,
   % style=authoryear-icomp,
    %sortlocale=de_DE,
    %natbib=true,
    %author=true,
%style=verbose,
%journal=true,
%url=true, 
%    doi=false,
%    eprint=true
]{biblatex}
\addbibresource{biblio.bib}

\usepackage[]{hyperref}
\hypersetup{
    colorlinks=true,
}


\newtheorem*{thm}{Theorem}
\newtheorem*{half}{Halfspace Condition}
%\newtheorem{lem}[thm]{Lemma}
%\newtheorem{prop}[thm]{Proposition}
\newtheorem{prob}{Problem}
\newtheorem*{cor}{Corollary}
%\newtheorem{question}[thm]{Question}
\newtheorem*{flashbang}{FlashBang Principle}
\newtheorem*{Lefschetz}{Lefschetz theorem}
\newtheorem*{hyp}{Hypothesis}
\theoremstyle{definition}
%\newtheorem{dfn}[thm]{Definition}
%\newtheorem{exx}[thm]{Example}
\theoremstyle{remark}
%\newtheorem{rem}[thm]{Remark}
\newcommand{\fh}{\mathfrak{h}}
\newcommand{\bn}{\mathbf{n}}
\newcommand{\bC}{\mathbb{C}}
\newcommand{\bG}{\mathbb{G}}
\newcommand{\bR}{\mathbb{R}}
\newcommand{\fB}{\mathfrak{B}}
\newcommand{\bZ}{\mathbb{Z}}
\newcommand{\bQ}{\mathbb{Q}}
\newcommand{\bH}{\mathbb{H}}
\newcommand{\bq}{\bar{Q}[t]}
\newcommand{\bb}{\bullet}
\newcommand{\del}{\partial}
\newcommand{\sB}{\mathscr{B}}
\newcommand{\sE}{\mathscr{E}}
\newcommand{\mR}{\bR^\times_{>0}}
\newcommand{\conv}{\overbar{conv}}
\newcommand{\sub}{\del^c \psi(y)}
\newcommand{\subb}{\del^c \psi^c(x'')}
\newcommand{\hh}{\hookleftarrow}
\newcommand{\bD}{\mathbb{D}}
\newcommand{\Gm}{\mathbb{G}_m}
\newcommand{\uF}{\underline{F}}
\newcommand{\sC}{\mathscr{C}}
\newcommand{\bX}{\overline{X}^{BS/\bQ}}
\newcommand{\sT}{\mathscr{T}}
\newcommand{\sW}{\mathscr{W}}
\newcommand{\sZ}{\mathscr{Z}}

\begin{document}

%\title{On Einstein's Alleged Proof of Invariance of Spherical Light Waves in Special Relativity}
\title{}


\author{J. H. Martel}
\date{\today}
\email{jhmartel@protonmail.com}
\maketitle

\begin{abstract}

\end{abstract}

\tableofcontents

\section{Introduction}

Our research program emphasizes concrete constructions, explicit examples, and we always search for elementary and intelligble explanations for seemingly complex phenomena. Thus our research is based on ideas from classical mechanics, convex geometry, and linear algebra. We are especially interested in applications of optimal transport (OT) to algebraic topology and geometry. 


\section{How to Construct Spines/Souls from Singularities}
In our PhD thesis [ref] we developed new applications of optimal transport to algebraic topology by studying the topological properties of singularities of optimal transport plans. We applied these results to constructing spines of various $BG$ models, where $G$ is a discrete group with finite virtual cohomological dimension, e.g. arithmetic groups, mapping class groups, knot groups, etc.. This was long term project (6+ years) which originated during our Masters at UBC, where we were introduced to the so-called "well rounded retract" of the moduli space of flat unimodular tori. I was given the problem of constructing a geometric retract of the moduli space of flat abelian varieties (flat symplectic tori), and achieved very limited progress. When I arrived in Toronto to begin my PhD, I was very fortunate to attend a course on optimal transportation taught by Prof. R. J. McCann, and a further course on "self-dual lagrangians" taught by Prof. Nassif Ghoussoub.  [INCOMPLETE]






%We seek elementary and intelligible mathematical theories. Our emphasis is always on concrete structures, on positive explicit constructions, as opposed to the abstract, implicit, presumptive methods of many ``modern" mathematicians. Thus our research is much informed by classical physics and mechanics, and we especially are interested in foundational issues underlying Einstein's Special and General Relativity. Applications of Optimal Transport and Algebraic Topology is a persistent theme in our investigations.


\section{Dold-Thom, Sweepouts, and Optimal Transportation}
In the article [ref] we develop further applications of OT to algebraic topology, and especially the problem of constructing sweepouts and homology cycles via solutions to OT programs. Our starting point is a variation on Dold-Thom's theorem. If $X$ is topological space, and $G$ is finitely-generated abelian group, then the Dold-Thom group is the kernel $AG_0(X;G):=\ker(\epsilon)$ of the augmentation map $$\epsilon: G(X) \to G,$$ defined by $\epsilon(\sum' g_x .x)=\sum' g_x$. Therefore $AG_0(X;G)$ consists of all finitely-supported $G$-valued distributions on $X$ with zero net sum (i.e. mean zero). Here $\emptyset$ represents the constant $0$-valued distribution on $X$. The vacuum state $\emptyset$ serves as canonical basepoint on $AG_0(X)$. Again the vacuum state $\emptyset$ is independant of any basepoint on $X$. 

\begin{thm}[Dold-Thom]
The singular reduced homology functor $X\mapsto \tilde{H}_*(X;G)$ is naturally equivalent to the functor of $\emptyset$-pointed homotopy groups $X\mapsto \pi_*(AG_0(X;G), \emptyset)$.
\end{thm}

As corollary, Dold-Thom establishes the following: If $Y$ is a Moore space, e.g. $Y=\mathbb{S}^q$ is a $q$-sphere, then $AG_0(Y;G)$ is a model of an Eilenberg-Maclane classifying space $K(G,q)$. 

According to the obstruction methods of Steenrod, Eilenberg, Maclane, the Dold-Thom theorem expresses a natural equivalence between free homotopy classes $[X, AG_0(Y;G)]$ and singular cohomology groups $H^q(X;G)$. This leads us to the following topological problem: 

\begin{prob}
\label{dt1}
Construct and classify homotopically nontrivial continuous maps $$f: X \to AG_0(Y)$$ for given topological space $X$ and $q$-sphere $Y=\mathbb{S}^q$.

For example, construct and classify continuous maps $f: \Sigma^2_g \to AG_0(\mathbb{S}^1)$, where $\Sigma_g^2$ is a connected Riemann surface.
\end{prob}

Remark. If $f$ is a continuous map solving \ref{dt1}, then regular fibres $f^{-1}(pt)$, where $pt$ represents a distribution on $Y$, are cycles in $X$. The generic cycles are homologically nontrivial whenever $f$ is homotopically nontrivial, [ref].

%It is already interesting problem to construct nontrivial continuous maps $f: \Sigma_g^2 \to AG_0(\mathbb{S}^1; \bZ)$, where $\Sigma_g^2$ represents a closed genus $g$ orientable surface. 

%The nature of $AG_0$ implies that such continuous maps are ``multi-valued"  and with net neutral point images. 

Evidently the group $AG_0(Y)$ can be viewed as a discretized version of the subspace $H$ of $L^2(Y)$ consisting of all measurable functions $f$ satisfying $\int_Y 1. f=0$. Thus $H=1^{\perp}$ is the orthogonal complement of the subspace of constant functions, generated by indicator functions on bounded subsets. Likewise we are replacing the set of Borel-Radon measures $\mathscr{M}(X\times Y)_{\geq 0}$ with the set of signed Borel-Radon measures having zero total measure $\int 1. d\mu(x) =0$.

From the viewpoint of OT, it is necessary to first construct interesting costs $c$ between the a given source space $X$ and the space of distributions on spheres.

\begin{prob} 
(Discrete Version) Construct and classify geometric costs $c$ associated with correlating probability measure $\sigma$ on $X$ and probability measures $\tau$ on $AG_0(\mathbb{S}^q)$;

(Continuous Version) Construct and classify the natural costs $c$ associated with correlating probability measures $\sigma$ on $X$ and probability measures $\tau$ on the orthogonal subspace $H$? 
\end{prob}

To be more specific about the meaning of ``natural" costs: we say a cost is natural if it represents a reasonable interaction energy between a unit source mass at $x$ and a unit target mass at $y$. We allow the possibility -- as experience shows us -- of the interaction energy $c(x,y)$ depending on the relative positions of $x,y$ relative to the target measure $\tau$. In this case the interaction energy has a Machian type interpretation [ref] where the cost $c(x,y)$ is defined relative to the universe environment. For our applications, the geometer seeks costs which satisfy basic regularity assumptions, e.g. as developed by Gangbo--McCann [ref] or the standard text \cite{Vil1}. 



\subsection{Sweepouts and OT}

Fix a source space $(X, \sigma)$. Concretely we assume $X$ is Riemannian, with metric $g$, and source measure $\sigma=vol_X$ equal to canonical volume measure (modulo scalars). Let $Y$ be a $k$-dimensional target Riemannian space, with target measure $\tau<<vol_Y$. We assume $\int_X \sigma \geq \int_Y \tau$, and say the source is abundant with respect to the target. To begin the study of optimal transports from source to target, we require a choice of cost function $c: X\times Y \to \bR$. For our geometric applications, it is convenient to make the following assumptions on $c$: 

(A0)

(A1)

(A2)

(A3)

In our thesis [ref] we compared and contrasted the properties of attractive costs, e.g. $c=d^2/2$ when $Y\subset X$, with a repulsive costs. The recent article [McCann--Kim] has studied interaction potentials which have the form [ref, error?]. Another interesting alternative is via Weber's potentials $V$, as discussed by AKTA (c.f. W.E.Weber's completed works), where the energy interaction between particles $x,y$ depends on their relative velocities. 

%A priori it is difficult to construct costs $c$ between spaces $X,Y$ which exist in totally different universes. In otherwords, the transport has no transportation path between them, and the prices of sale $-\psi(y)$ and delivery $\psi^c(x)$ cannot be observed, except if there is some instaneous teleportation between the points of sale and delivery. 

Practically, the author finds the best results are obtained when the target $Y$ is given as a subset of the source, or say by some canonical embedding $Y\hookrightarrow X$. Interesting applications arise when $Y=\del X[t]$, where $X[t] \subset X$ is a ``rational excision" of $X$. [See ref, thesis].

%The attractive costs measure the interaction energies of oppositely charged particles on the source and target, e.g. when source has all negative charges, and target has exclusively positive charges. The quadratic cost $c=d^2/2$ is a type of attractive cost. The repulsive costs represent the interaction energies of negative source charges with negative target charges. 

Given a cost $c$ satisfying the assumptions (A), we obtain a contravariant functor $Z: 2^Y \to 2^X$ defined by $Z(Y_I):=\cap_{y\in Y_I} \sub$, where $\psi=\psi^{cc}$ is the Kantorovich potential maximizing the dual program. Our hypotheses on $c$ imply the cells $Z(y)$ are $(n-k)$-dimensional cycles in $X$ for almost every point $y\in Y$.. In other words, the OT program defined by the data $(\sigma, \tau, c)$ generates -- via Kantorovich duality -- a \emph{nontrivial contravariant functor} $Z: 2^Y \to 2^X$. And as we vary $y$ over the target $(Y, \tau)$, restricting the functor $Z$ to the singletons $Y \hookrightarrow 2^Y$, we obtain $Y$-parameter measurable family of closed subsets $Z(y)$ on $X$. Our proposal is that $y\mapsto Z(y)$ is an interesting ``measurable" sweepout of $X$. In many cases we find these maps are actually topological sweepouts, and continuous in the necessary topologies. The continuity is defined with respect to Almgren's flat topology. Explicitly, the continuity of $y\mapsto Z(y)$ requires proving the following: if $y_0, y_1$ are sufficiently close in $Y$, then the cycles $Z(y_0)$ and $Z(y_1)$ bound a Lipschitz chain of small area, i.e. there exists a chain $C$ such that $\del C = Z(y_1)-Z(y_0)$ and $C$ has arbitrarily small area.


\begin{prob}
\label{alm}
Given a source space $(X, \sigma)$, study costs $c$ and target spaces $(Y, \tau)$ for which the $Y$-parameter family of subsets $Z(y)$ defines topological sweepouts of the source. More concretely, under the assumptions (A), prove the $Y$-parameter family of cycles $Z$ is continuous in Almgren's flat topology on chains. 
\end{prob}
%A positive solution of Problem \ref{alm} implies the $Y$-parameter family of cycles can be viewed as a continuous topological sweepout of $X$. This also requires proving that the cycles $Z(y)$, $y\in Y$, assemble to the fundamental class $[X]$ of the source space. 

Thus we propose using the regularity theory of OT to generate continuous topological objects from the a\` priori measurable objects arising from Monge-Kantorovich duality. 

\subsection{Application to Guth's Width Inequalities}

The previous section introduced the possibility of constructing topological sweepouts via solutions of OT programs. Now we study the possibility of representing minimal sweepouts by such solutions. For applications, we were motivated by Guth-Gromov's width inequality [ref]: 

\begin{quote} 
\emph{ If $M^n$ is a closed Riemannian manifold, then there exists a universal constant $C(n)$ depending only on the dimension $n$ such that $width_k(X,g)^{1/k}\leq C(n) vol(X,g)^{1/n}.$
} 
\end{quote} 



We recall that the $k$-width is defined by a min-max problem, namely $$width_k(X,g):=\min_{\{z_t\}} \max_{t} vol_k(z_t),$$ where the minimum ranges over all $k$-parameter sweepouts $z$ of $X$. Estimates on $width_k$ imply every $k$-sweepout contains at least one cycle of large volume. Now our task is to interpret $width_k$ in terms of optimal transportation.  Let $\sigma, \tau, c$ be as above, with $Z(y):=\sub$. The assumptions (A) imply the existence of a measurable map $T: X\to Y$ defined $\sigma$-a.e. satisfying $T\# \sigma = \tau$, and such that $$g(y)=\int_{T^{-1}(y)} \frac{1}{|JT|} f(x) d\mathscr{H}^{n-k}(x) $$ for $\tau$-a.e. $y\in Y$. The term $JT$ is a type of Jacobian term, e.g. $|JT|=|\det {}^t DT DT|^{1/2}$. Now the fibre volume is evidently $$vol_{n-k}[T^{-1}(y)]=vol_{n-k} [\sub]=\int_{T^{-1}(y)} 1 . f(x) d\mathscr{H}^{n-k}(x).$$ Trivially, if the derivative $DT$ had constant magnitude along the fibres $T^{-1}(y)$, then we could immediately compare the density $g(y)=d\tau/d\mathscr{H}^{k}|_y$ with the $(n-k)$-volume of the fibre $T^{-1}(y)$. Naively we might expect the density $g(y)$ to be exactly equal to the fibre volume, as we find in the discrete case where $\dim(Y)=0$. But in general the Jacobian-type term $|JT|$ is nontrivial. 

We observe that the width of the sweepout only depends on the \emph{maximal} volume fibre. [INCOMPLETE]

% How to estimate the maximal volume? Is it easier to compute the average volume of the fibres? But we only have trivial lower bound for MAX compared to AVG. Can we use the trick from " Intrinsic Volumes of Level Sets"? 


Let us consider the above problem with one-dimensional target. Then we find $T: X \to Y$ is a measurable function, and the above formulas become 
$$g(y)=\int_{T^{-1}(y)} \frac{1}{||DT||}, $$ where $||DT||$ is the square root of the sum of squares of $DT$, i.e. the vector magnitude of $\nabla T$. The problem is to relate the integral formula for $g(y)$ with $vol(T^{-1}(y))$, and to determine the maximal(!) fibre volume. Note: we have no way of varying the functional $y\mapsto vol(T^{-1}(y))$, e.g. to identify the local extrema.


Remarks. The fibres $T^{-1}(y)$, $y\in Y$, are not generic fibres of a generic measurable map pushing $\sigma$ forward to $\tau$. Kantorovich duality implies $T$ has further geoetric convex properties. For instance, the fibre $T^{-1}(y)$ essentially consists of points $x\in X$ for which equality holds in $$-\phi(x)+\psi(y) \leq c(x,y), $$ where $\phi, \psi$ are the $c$-dual maximizing solutions to the Kantorovich program. If we differentiate the equality $-\phi(x)+\psi(y)=c(x,y)$, for $x\in T^{-1}(y)$, with respect to the source variable $x$, then we find $$D_x c + D_x \phi=0$$ and $$D^2_{xx}(c+\phi) \geq 0.$$ In otherwords $T^{-1}(y)$ appears as the global minimizing set of a locally semiconvex function $x\mapsto c(x,y) +\phi(x)$.  Now consider further differentiating the identity $$D_x c(x, T(x)) + D_x \phi(x)=0$$ with respect to $x$. We obtain $$-D^2_{xx} (c+\phi)=D^2_{xy} c \cdot DT.$$ If we assume $D^2_{xy}c$ has maximal rank everywhere, then we can find a left inverse for $D^2_{xy}c$, namely $({}^t D^2_{xy} c D^2_{xy} c)^{-1} {}^t D^2_{xy} c$, and thus we obtain 
\begin{equation}\label{a} ({}^t D^2_{xy} c D^2_{xy} c)^{-1} \cdot {}^t D^2_{xy} c \cdot (-D^2_{xx} c +\phi) =DT
\end{equation} along the fibre $x\in T^{-1}(y)$. 
 
When the target $Y$ is one-dimensional, then \eqref{a} takes the form 
\begin{equation}\label{a1}
DT=\frac{-1}{||\nabla_x \frac{\partial c}{\partial y}||^2} D^2_{xx}(c+\phi) \cdot \nabla_x \frac{\partial c}{\partial y}   
\end{equation} along the fibre $T^{-1}(y)$. 




%This idea took many years to discover, and the author remembers being perplexed during his Masters on \emph{how} to construct small cycles on surfaces, and tori, and abelian varieties, etc.. Also interesting were the problems of systolic geometry, for example Guth and Gromov's width inequalities: 






\section{Souls and Spines}
Many years ago a professor (J.S) told me about spines of $PGL(\mathbb{Z}^n)$, and encouraged me to try and construct spines for $Sp(\bZ^{2g}, \omega)$ and various other discrete groups, e.g. arithmetic groups, and mapping class groups of Riemann surfaces. A method for constructing spines, and producing a wide variety of candidate spines, was the subject of our PhD thesis [ref]. Our idea was to exhibit the spines in the singularities (locus of discontinuity) of an OT program, and this reduction to singularity was the main topological tool developed via the regularity theory of OT. Beyond the construction of small dimensional classifying spaces, which was our original motivation, the construction of spines, or souls, as the singularity of an optimal transport map appears to have wider applications. We discuss some extensions of our ideas below.

\subsection{Medial Axis Transform}

Blum's medial axis transform $M(A)$ of an open subset $A\subset \bR^n$...


\subsection{Souls in Singular Alexandrov Spaces}
Let $(X,d)$ be an open complete finite-dimensional Alexandrov space with nonnegative sectional curvature $\kappa \geq 0$. We do not assume $X$ is everywhere regular, i.e. that $X$ is a manifold, and allow the possibility that $X$ is ``singular" in the terminology of the Alexandrov synthetic geometers. For such geometers (which the author is not) the problem of constructing souls $S$ of $X$ is the problem of finding compact totally geodesic subspaces $S \hookrightarrow X$ such that $X$ strongly deformation retracts onto $S$. Notice the hypothesis on geodesic convexity of $S$ via path geodesics. When $X$ is smooth manifold the theory is well developed, e.g. G. Perelman [ref] proved that $X$ is homeomorphic to a disk bundle over $S$. The construction of $S$, in the case of smooth manifolds, is due to Cheeger-Gromoll [ref] using a construction of Sharafutdinov [ref]. 

Our goal is to develop the non smooth theory of Alexandrov Souls as an application of the Reduction to Singularity ideas of our thesis [ref]. The arguments of our thesis depend on choices of source measures, target measures, and costs $c$. Let us summarize the construction: we are given an Alexandrov space $(X,d)$ satisfying the above hypotheses. If $X$ is not simply connected, then replace with a universal covering space $X^\tilde$.




\section{Mapping Class Groups}
\subsection{Closing the Steinberg Symbol}

\section{}



\section{Special Relativity}
\subsection{Lorentz Non Invariance of Spherical Light Waves}
Our research [ref] has identified a critical error in the foundations of Einstein's special relativity. The error essentially consists in this: that ``radius" is not Lorentz invariant variable, e.g. the class of radial solutions of the homogeneous wave equation is not invariant under Lorentz transformations. This is related to a gap in Einstein's attempt to reconcile the principle of relativity (applied to non accelerated inertial frames $K, K'$) with the law of propapagation of light. In simplest terms, the hypothesis that spherical light waves remain spherical in every inertial frame $K$, is untenable, and not a Lorentz invariant property. The error, present in almost every treatment of Einstein's special theory, is easily overlooked when one mistakes the expression $$\xi^2+\eta^2+\zeta^2=c^2 \tau^2$$ with the equation of a sphere. In itself, such an expression, which is indeed Lorentz invariant, represents only a null cone. To specify a sphere requires a second equation, for example, $$d(\xi^2+\eta^2+\zeta^2)=0.$$ But then we observe that the expression $\xi^2+\eta^2+\zeta^2$ and its differential, is definitely \emph{not} Lorentz invariant. Again, this is critical error at the basis of SR.

\subsection{R. Sansbury's experiment: Does Light Travel Through Space?}
 
\section{General Relativity}


\subsection{Einstein's Topological PCA versus Gromov's Probabilistic PCA}
For Einstein, the only objective reality is the topological coincidence of points. This is the motivation for his seeking tensor equations, wherein the zeros $a_{ij}=0$ are invariant with respect to change of variable, as the basis for a general theory of gravitation. For example, the left hand side of Einstein's field equations $$R_{ij}-R g_{ij}=\kappa T_{ij}$$ is evidently a tensor quantity, since the Ricci tensor $Ric=(R_{ij})$ is indeed tensorial, as is its trace $R$ (scalar curvature). In our research [ref] we investigate a probabilistic formulation of Einstein's PCA, which is much closer to Gromov's category of finite probability spaces. In this work, we replace ``tensor equations" -- which in our mind are much too idealized objects over the real numbers $\bR$ --  with stochastic ``frequency equations". 

Idea: the basic tensor expression $$d\xi=\frac{\partial \xi}{\partial x_1} dx_1+\cdots + \frac{\partial \xi}{\partial x_n} dx_n$$ has a probabilistic interpretation when we replace $d\xi$ with the time-rate $d\xi/dt$. [Compare Gromov's paper on Mendel, where tensors and distributions are identified].

\subsection{``Is Energy Well-Defined in GR?"}
The title of this subsection is taken from an unpublished essay of Erik Curiel. The right hand side of Einstein's field equations $\kappa T_{ij}$ describes the contribution of matter (or ``mass-energy") to the system, including the electromagnetic energies. See models for $T_{ij}$ representing a perfect fluid, or electric field, etc.. However Einstein's equivalence principle views gravitional potential energy as coincident (in an accelerated frame) with the motion or inertial mass of an object. [INCOMPLETE]

- What are the physical units in Einstein's field equations?

- Does general covariance permit the use of physical units?

- In stable low energy conditions, the map $$\text{matter} \to \text{mass}$$ is constant, and the conservation of matter (which can neither be created nor destroyed) manifests as the conservation of mass. In this stable setting, one is justified to interpret the stress-energy tensor $T_{ij}$ as satisfying a continuity equation. In this case, we then find $T_{ij}$ is divergence free.

\subsection{GR and Gravitational Waves}

\subsection{GR and Thermodynamics}

The role of physical coordinates in GR is very controversial. Since the work of Hilbert [ref], mathematicians have developed the hazardous habit of discarding apparent singularities as ``merely artifacts of the given coordinate system". However thermodynamics contains strict rules for evaluating and comparing physical units. E.g., it relatives temperature to average kinetic energy. And indeed, the theory distinguishes kinetic energy from potential energy.

- is general covariance incompatible with the existence of objective physical units (e.g. temperature and thermodynamics)?


\subsection{GR and Black Holes}






\printbibliography[title={References}]
\end{document}
