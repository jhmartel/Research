\documentclass[12pt]{amsart}
\usepackage{mathrsfs}
\usepackage{amsmath}
\usepackage{amssymb}
\usepackage{amsfonts}
\usepackage{amsopn}
\usepackage{amsthm}
\usepackage{latexsym}
\usepackage[all]{xy}
\usepackage{enumerate}
\usepackage{geometry}
%\usepackage{biblatex}
%\usepackage{hyperref}
%\usepackage[autostyle]{csquotes}
\usepackage{fancyhdr}
\usepackage{graphicx}
\usepackage{wrapfig}
\usepackage{float}

\usepackage[
    backend=biber,
style=alphabetic,
sorting=nyt,
   % style=authoryear-icomp,
    %sortlocale=de_DE,
    %natbib=true,
    %author=true,
%style=verbose,
%journal=true,
%url=true, 
%    doi=false,
%    eprint=true
]{biblatex}
\addbibresource{biblio.bib}

\usepackage[]{hyperref}
\hypersetup{
    colorlinks=true,
}


\newtheorem*{thm}{Theorem}
\newtheorem*{half}{Halfspace Condition}
%\newtheorem{lem}[thm]{Lemma}
%\newtheorem{prop}[thm]{Proposition}
\newtheorem{prob}{Problem}
\newtheorem*{cor}{Corollary}
%\newtheorem{question}[thm]{Question}
\newtheorem*{flashbang}{FlashBang Principle}
\newtheorem*{Lefschetz}{Lefschetz theorem}
\newtheorem*{hyp}{Hypothesis}
%\theoremstyle{definition}
\newtheorem{dfn}{Definition}
%\newtheorem{exx}[thm]{Example}
\theoremstyle{remark}
%\newtheorem{rem}[thm]{Remark}
\newcommand{\fh}{\mathfrak{h}}
\newcommand{\bn}{\mathbf{n}}
\newcommand{\bC}{\mathbb{C}}
\newcommand{\bG}{\mathbb{G}}
\newcommand{\bR}{\mathbb{R}}
\newcommand{\fB}{\mathfrak{B}}
\newcommand{\bZ}{\mathbb{Z}}
\newcommand{\bQ}{\mathbb{Q}}
\newcommand{\bH}{\mathbb{H}}
\newcommand{\bq}{\bar{Q}[t]}
\newcommand{\bb}{\bullet}
\newcommand{\del}{\partial}
\newcommand{\sB}{\mathscr{B}}
\newcommand{\sE}{\mathscr{E}}
\newcommand{\mR}{\bR^\times_{>0}}
\newcommand{\conv}{\overbar{conv}}
\newcommand{\sub}{\del^c \psi(y)}
\newcommand{\subb}{\del^c \psi^c(x'')}
\newcommand{\hh}{\hookleftarrow}
\newcommand{\bD}{\mathbb{D}}
\newcommand{\Gm}{\mathbb{G}_m}
\newcommand{\uF}{\underline{F}}
\newcommand{\sC}{\mathscr{C}}
\newcommand{\bX}{\overline{X}^{BS/\bQ}}
\newcommand{\sT}{\mathscr{T}}
\newcommand{\sW}{\mathscr{W}}
\newcommand{\sZ}{\mathscr{Z}}

\begin{document}

%\title{On Einstein's Alleged Proof of Invariance of Spherical Light Waves in Special Relativity}
\title{}


\author{J. H. Martel}
\date{\today}
\email{jhmartel@protonmail.com}
\maketitle

\begin{abstract}

\end{abstract}

\tableofcontents

\section{Introduction}
We seek elementary and intelligible mathematical theories. Our emphasis is always on concrete structures, on positive explicit constructions, as opposed to the abstract, implicit, presumptive methods of many ``modern" mathematicians. Thus our research is much informed by classical physics and mechanics, and we especially are interested in foundational issues underlying Einstein's Special and General Relativity. Applications of Optimal Transport and Algebraic Topology is a persistent theme in our investigations.

\section{Special Relativity}
\subsection{Lorentz Non Invariance of Spherical Light Waves}
Our research [ref] has identified a critical error in the foundations of Einstein's special relativity. The error essentially consists in this: that ``radius" is not Lorentz invariant variable, e.g. the class of radial solutions of the homogeneous wave equation is not invariant under Lorentz transformations. This is related to a gap in Einstein's attempt to reconcile the principle of relativity (applied to non accelerated inertial frames $K, K'$) with the law of propapagation of light. In simplest terms, the hypothesis that spherical light waves remain spherical in every inertial frame $K$, is untenable, and not a Lorentz invariant property. The error, present in almost every treatment of Einstein's special theory, is easily overlooked when one mistakes the expression $$\xi^2+\eta^2+\zeta^2=c^2 \tau^2$$ with the equation of a sphere. In itself, such an expression, which is indeed Lorentz invariant, represents only a null cone. To specify a sphere requires a second equation, for example, $$d(\xi^2+\eta^2+\zeta^2)=0.$$ But then we observe that the expression $\xi^2+\eta^2+\zeta^2$ and its differential, is definitely \emph{not} Lorentz invariant. Again, this is critical error at the basis of SR.

\subsection{R. Sansbury's experiment: Does Light Travel Through Space?}
 
\section{General Relativity}


\subsection{Einstein's Topological PCA versus Gromov's Probabilistic PCA}
For Einstein, the only objective reality is the topological coincidence of points. This is the motivation for his seeking tensor equations, wherein the zeros $a_{ij}=0$ are invariant with respect to change of variable, as the basis for a general theory of gravitation. For example, the left hand side of Einstein's field equations $$R_{ij}-R g_{ij}=\kappa T_{ij}$$ is evidently a tensor quantity, since the Ricci tensor $Ric=(R_{ij})$ is indeed tensorial, as is its trace $R$ (scalar curvature). In our research [ref] we investigate a probabilistic formulation of Einstein's PCA, which is much closer to Gromov's category of finite probability spaces. In this work, we replace ``tensor equations" -- which in our mind are much too idealized objects over the real numbers $\bR$ --  with stochastic ``frequency equations". 

Idea: the basic tensor expression $$d\xi=\frac{\partial \xi}{\partial x_1} dx_1+\cdots + \frac{\partial \xi}{\partial x_n} dx_n$$ has a probabilistic interpretation when we replace $d\xi$ with the time-rate $d\xi/dt$. [Compare Gromov's paper on Mendel, where tensors and distributions are identified].

\subsection{``Is Energy Well-Defined in GR?"}
The title of this subsection is taken from an unpublished essay of Erik Curiel. The right hand side of Einstein's field equations $\kappa T_{ij}$ describes the contribution of matter (or ``mass-energy") to the system, including the electromagnetic energies. See models for $T_{ij}$ representing a perfect fluid, or electric field, etc.. However Einstein's equivalence principle views gravitional potential energy as coincident (in an accelerated frame) with the motion or inertial mass of an object. [INCOMPLETE]

- What are the physical units in Einstein's field equations?

- Does general covariance permit the use of physical units?

- In stable low energy conditions, the map $$\text{matter} \to \text{mass}$$ is constant, and the conservation of matter (which can neither be created nor destroyed) manifests as the conservation of mass. In this stable setting, one is justified to interpret the stress-energy tensor $T_{ij}$ as satisfying a continuity equation. In this case, we then find $T_{ij}$ is divergence free.

\subsection{GR and Gravitational Waves}

\subsection{GR and Thermodynamics}

The role of physical coordinates in GR is very controversial. Since the work of Hilbert [ref], mathematicians have developed the hazardous habit of discarding apparent singularities as ``merely artifacts of the given coordinate system". However thermodynamics contains strict rules for evaluating and comparing physical units. E.g., it relatives temperature to average kinetic energy. And indeed, the theory distinguishes kinetic energy from potential energy.

- is general covariance incompatible with the existence of objective physical units (e.g. temperature and thermodynamics)?


\subsection{GR and Black Holes}



\section{Sweepouts and Optimal Transportation}
Henceforth we assume $X$ is a complete finite dimensional Riemannian manifold, possibly noncompact, with Riemannian metric $g$, and source measure $\sigma$ on $X$ proportional to the volume measure $vol_X$. For applications we suppose $X$ is a fixed source space. Let $Y$ be a $k$-dimensional target Riemannian space, with target measure $\tau<<vol_Y$. We assume $\int_X \sigma \geq \int_Y \tau$, and say the source is abundant with respect to the target. We allow the target $Y$ to be variable, as we discuss below.



\subsection{Dold-Thom Theorem}
In the article [ref] we develop further applications of OT to algebraic topology, and especially the problem of constructing sweepouts and homology cycles via solutions to OT programs. Our starting point is a version of Dold-Thom's theorem. 

\begin{dfn} If $X$ is topological space, and $G$ is finitely-generated abelian group, then the Dold-Thom group is the kernel $AG_0(X;G):=\ker(\epsilon)$ of the augmentation map $\epsilon: G(X) \to G,$ defined by $\epsilon(\sum' g_x .x)=\sum' g_x$. 
\end{dfn}

Therefore $AG_0(X;G)$ consists of all finitely-supported $G$-valued distributions on $X$ with zero net sum (zero net charge). Here $\emptyset$ represents the constant $0$-valued distribution on $X$. The vacuum state $\emptyset$ serves as canonical basepoint on $AG_0(X)$.

\begin{thm}[Dold-Thom]
The singular reduced homology functor $X\mapsto \tilde{H}_*(X;G)$ is naturally equivalent to the functor of $\emptyset$-pointed homotopy groups $X\mapsto \pi_*(AG_0(X;G), \emptyset)$.
\end{thm}

\begin{cor}
If $Y$ is a Moore space, e.g. $Y=\mathbb{S}^q$ is a $q$-sphere, then $AG_0(Y;G)$ is a model of an Eilenberg-Maclane classifying space $K(G,q)$.
\end{cor}

According to the obstruction methods of Steenrod, Eilenberg, Maclane, it follows that there is a natural equivalence between free homotopy classes $[X, AG_0(Y;G)]$ and singular cohomology groups $H^q(X;G)$. Thus Dold-Thom allows the construction of cohomology cycles via the construction of topologically nontrivial maps $X\to AG_0(Y;G)$ whenever $H^q(X;G)\neq 0$. This leads us to a general topological problem: 

\begin{prob}
\label{dt1}
Construct and classify homotopically nontrivial continuous maps $$f: X \to AG_0(Y)$$ for given topological space $X$ and $q$-sphere $Y=\mathbb{S}^q$ whenever $H^q(X)\neq 0$. 

For example, construct and classify continuous maps $f: \Sigma^2_g \to AG_0(\mathbb{S}^1)$, where $\Sigma_g^2$ is a connected Riemann surface.
\end{prob}

Maps $f$ solving Problem \ref{dt1} generate homological cycles on $X$. The argument is described in R. Kirby's book \cite{kirby}. If $f$ is a continuous map solving \ref{dt1}, then regular fibres $f^{-1}(pt)$, where $pt$ represents a distribution on $Y$, are cycles in $X$. These cycles are nontrivial whenever $f$ is homotopically nontrivial, and even Poincar\'e dual to the cocycle generated by $f$ (assuming $X$ is compact oriented, and $Y$ is a Moore space).

%The nature of $AG_0$ implies that such continuous maps are ``multi-valued"  and with net neutral point images. 

%The topological abelian group $AG_0(Y)$ is a discretization of the subspace $H$ of $L^2(Y)$ consisting of measurable functions $f$ satisfying $\int_Y 1. f=0$. Thus $H=1^{\perp}$ is the orthogonal complement to the subspace of constant functions on $Y$. 

For applications of optimal transport, it is necessary to first construct interesting costs $c$ between a given source space $X$ and the space of distributions on spheres. 

\begin{prob} 
Construct and classify natural geometric costs $c$ associated with correlating probability measure $\sigma$ on $X$ and probability measures $\tau$ on $AG_0(\mathbb{S}^q)$.

%(Continuous Version) Construct and classify the geometric costs $c$ associated with correlating probability measures $\sigma$ on $X$ and probability measures $\tau$ on the orthogonal subspace $H$? 
\end{prob}

What is a natural geometric cost? We say a cost is natural if it represents a reasonable interaction energy between a unit source mass at $x$ and a unit target mass at $y$. We allow the possibility -- as experience shows us -- of the interaction energy $c(x,y)$ depending on the relative positions of $x,y$ relative to the target measure $\tau$. In this case the interaction energy has a Machian type interpretation. The costs should also satisfy some basic regularity assumptions. See [ref]. To fix ideas, it is already interesting problem to construct interesting costs between a surface $\Sigma_g^2$ and, say, $AG_0(S^1)$.



\subsection{Sweepouts and OT}

To begin the study of optimal transport from source $(X,\sigma)$ to targets $(Y,\tau)$, we require a choice of cost function $c: X\times Y \to \bR$. For geometric applications it is convenient to make the following assumptions on $c$: 

\begin{itemize}
\item[\textbf{(A0)}]\emph{(Continuous, positive, coercive)} The cost $c$ is continuous on $dom(c) \subset X\times Y$, and nonnegative $c(x,y)\geq 0$. Moreover we assume $x\mapsto c(x,y)$ is \emph{coercive} for every $y\in Y$, i.e. the sublevels $\{x\in X~|~ c(x,y) \leq t\}$ are compact subsets of $X$ for every $t\in \bR$, and $y\in Y$.  


\item[\textbf{(A1)}]\emph{(Uniformly $C^2$ in source variable)} The cost is twice continuously differentiable with respect to the source variable $x$, uniformly in $y$ throughout $dom(c)$. So for every $y\in Y$, the Hessian $x\mapsto \nabla_{xx}^2 c(x,y)$ exists and is continuous throughout $dom(c(\cdot, y))$. 

\item[\textbf{(A2)}]\emph{(Jointly weak $C^1$)} The function $(x,y)\mapsto ||\nabla_x c(x,y)||$ is upper semicontinuous throughout $dom(c)$. So for every $t\in \bR$ the superlevel set $\{ ||\nabla_x c(x,y)|| \geq t\}$ is a closed subset of $dom(c)$. 

\item[\textbf{(A3)}]\emph{(Nonconstant on open subsets)} For every $y\in Y$, we assume $x'\mapsto \nabla_x c(x',y)$ does not vanish identically on any open subset of $dom(c_y)$.  

%For every Radon measure $\sigma$ on the source $X$ and $y\in Y$, the single-variable function $t \mapsto \sigma[\{c_y< t\}]$ is strictly monotone-increasing for $t\in supp(c_y\# \sigma) \subset \bR$. 

% The cost satisfies \textbf{(Mono)}\label{mon} A cost $c: X\times Y\to \bR\cup \{+\infty\}$ is monotone with respect to a source measure $\sigma$ if for every $y\in Y$, 

\item[\textbf{(A4)}]\emph{(Twisted on source)} The cost satisfies (Twist) condition with respect to the source variable throughout $dom(c)$: for every $x'\in X$ the rule $y\mapsto \nabla_x c(x',y)$ defines an injective mapping $dom(c_{x'}) \to T_{x'} X$. 

\item[\textbf{(A5)}]\emph{($C^1$ in target variable)}\label{A+} For every $x\in X$, the function $y\mapsto c(x,y)$ is continuously differentiable; and for every $y\in Y$, the gradients $\nabla_y c(x,y)$ are bounded on compact subsets $K\subset X$.
\end{itemize}

For simplicity, the reader can assume that $c\in C^2(X\times Y)$ and that (Twist) is satisfed on source (A4).  

In our thesis [ref] we compared the properties of \emph{attractive} costs, e.g. the quadratic geodesic cost $c(x,y)=d(x,y)^2/2$ when $Y\subset X$, with the class of so-called \emph{repulsive costs}. Heuristically, the attractive costs represents interaction energies between oppositely charged positive source and and negatively charged target configurations. The repulsive cost represents interaction energies between, say, positive source and positive target configurations. We recall that ``opposite charges attract" and "like charges repel", hence the terminology. 

%The recent article [McCann--Kim] has studied interaction potentials which have the form. Another interesting alternative is via Weber's potentials $V$, as discussed by AKTA (c.f. W.E.Weber's completed works).

\`A priori, it is difficult to construct costs $c$ between spaces $X, Y$ which occupy different spaces, and which have no spatial relations. Without a transport path between points $x,y$ it is hard to relate the price of sale $-\psi(y)$ and price of purchase $\psi^c(x)$. Practically speaking, the author finds the best results are obtained when the target $Y$ is given as a subset of the source, or say by some canonical embedding $Y\hookrightarrow X$. Most interesting applications arise when $Y=\del X[t]$, where $X[t] \subset X$ is a ``rational excision" of $X$ (see our thesis [ref:chapter] for illustrations). 


Now suppose we have a cost $c$ satisfying the assumptions (A). The OT program defined by the data $(\sigma, \tau, c)$ generates via Kantorovich duality a \emph{nontrivial contravariant functor} $Z: 2^Y \to 2^X$ defined by $Z(Y_I):=\cap_{y\in Y_I} \sub$, where $\psi=\psi^{cc}$ is the Kantorovich potential maximizing the dual program. As we vary $y$ over the target $(Y, \tau)$ -- restricting the functor $Z$ to the singletons $Y \hookrightarrow 2^Y$ -- we obtain $Y$-parameter family of closed subsets $Z(y)$ on $X$. Our hypotheses on $c$ imply the cells $Z(y)$ are $(n-k)$-dimensional subvarieties in $X$ for almost every point $y\in Y$ [ref]. We propose that the parameterization $y\mapsto Z(y)$ defines topological sweepouts, and which are continuous in the necessary topologies. Explicitly, this requires proving: if $y_0, y_1$ are sufficiently close in $Y$, then the cycles $Z(y_0)$ and $Z(y_1)$ bound a Lipschitz chain of small area, i.e. there exists a chain $C$ such that $\del C = Z(y_1)-Z(y_0)$ and $C$ has arbitrarily small area. This also requires proving that the cycles $Z(y)$, $y\in Y$, assemble to the fundamental class $[X]$ of the source space. 


\begin{prob}
\label{alm}
Given a source space $(X, \sigma)$, construct and study costs $c$ and target spaces $(Y, \tau)$ for which the $Y$-parameter family of subsets $y\mapsto Z(y)$ defines topological sweepouts of the source $X$. More concretely, under the assumptions (A), prove the $Y$-parameter family of cycles $Z$ is continuous in Almgren's flat chain topology. 
\end{prob}

A positive solution of Problem \ref{alm} implies the $Y$-parameter family of cycles can be viewed as a continuous topological sweepout of $X$. Thus we propose using the regularity theory of OT to generate continuous topological objects from the measure theoretic objects arising from Monge-Kantorovich duality. 

\subsection{Application to Guth's Width Inequalities}

The previous section introduced the possibility of constructing topological sweepouts via solutions of OT programs. Now we study the possibility of representing minimal sweepouts by such solutions. For applications, we were motivated by Guth-Gromov's width inequality [ref]: 

\begin{quote} 
\emph{ If $M^n$ is a closed Riemannian manifold, then there exists a universal constant $C(n)$ depending only on the dimension $n$ such that $width_k(X,g)^{1/k}\leq C(n) vol(X,g)^{1/n}.$
} 
\end{quote} 

We recall that the $k$-width is defined by a min-max problem, namely $$width_k(X,g):=\min_{\{z_t\}} \max_{t} vol_k(z_t),$$ where the minimum ranges over all $k$-parameter sweepouts $z$ of $X$. Estimates on $width_k$ imply every $k$-sweepout contains at least one cycle of large volume. Now our task is to interpret $width_k$ in terms of optimal transportation.  Let $\sigma, \tau, c$ be as above, with $Z(y):=\sub$. The assumptions (A) imply the existence of a measurable map $T: X\to Y$ defined $\sigma$-a.e. satisfying $T\# \sigma = \tau$, and such that $$g(y)=\int_{T^{-1}(y)} \frac{1}{||DT||} f(x) d\mathscr{H}^{n-k}(x) $$ for $\tau$-a.e. $y\in Y$. Evidently $$vol_{n-k}[T^{-1}(y)]=vol_{n-k} [\sub]=\int_{T^{-1}(y)} 1 . f(x) d\mathscr{H}^{n-k}(x).$$ Now trivially, if the derivative $DT$ had constant magnitude along the fibres $T^{-1}(y)$, then we could immediately compare the density $g(y)=d\tau/d\mathscr{H}^{k}|_y$ with the $(n-k)$-volume of the fibre $T^{-1}(y)$. However the width of the sweepout only depends on the fibre of \emph{maximal} $(n-k)$ volume. [INCOMPLETE].









%This idea took many years to discover, and the author remembers being perplexed during his Masters on \emph{how} to construct small cycles on surfaces, and tori, and abelian varieties, etc.. Also interesting were the problems of systolic geometry, for example Guth and Gromov's width inequalities: 






\section{Souls and Spines}
Many years ago a professor (J.S) told me about spines of $PGL(\mathbb{Z}^n)$, and encouraged me to try and construct spines for $Sp(\bZ^{2g}, \omega)$ and various other discrete groups, e.g. arithmetic groups, and mapping class groups of Riemann surfaces. A method for constructing spines, and producing a wide variety of candidate spines, was the subject of our PhD thesis [ref]. Our idea was to exhibit the spines in the singularities (locus of discontinuity) of an OT program, and this reduction to singularity was the main topological tool developed via the regularity theory of OT. Beyond the construction of small dimensional classifying spaces, which was our original motivation, the construction of spines, or souls, as the singularity of an optimal transport map appears to have wider applications. We discuss some extensions of our ideas below.

\subsection{Medial Axis Transform}

Blum's medial axis transform $M(A)$ of an open subset $A\subset \bR^n$...


\subsection{Souls in Singular Alexandrov Spaces}
Let $(X,d)$ be an open complete finite-dimensional Alexandrov space with nonnegative sectional curvature $\kappa \geq 0$. We do not assume $X$ is everywhere regular, i.e. that $X$ is a manifold, and allow the possibility that $X$ is singular. The problem of constructing souls $S$ of $X$ is the problem of finding a compact totally geodesic subspace $S \hookrightarrow X$ such that $X$ strongly deformation retracts onto $S$. When $X$ is smooth manifold, then G. Perelman [ref] proved that $X$ is homeomorphic to a disk bundle over $S$. The construction of $S$, in the case of smooth manifolds, is due to Cheeger-Gromoll [ref] and Sharafutdinov [ref]. 





\section{Mapping Class Groups}
\subsection{Closing the Steinberg Symbol}

\section{}




\printbibliography[title={References}]
\end{document}
